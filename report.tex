\documentclass[12pt,paper=a4]{scrartcl}
\usepackage[T1]{fontenc}        % Sorgt u.a. dafür, dass Texte vernünftig markierbar werden auch bei Sonderzeichen
\usepackage[utf8]{inputenc}	% Wir wollen UTF-8 (=keine Probleme mit Umlauten etc.)
\usepackage{ae,aecompl} %bessere Schrift
\usepackage{amsthm}
\usepackage{amsmath}
\usepackage{amssymb}
\usepackage{caption}
\captionsetup[figure]{labelformat=empty}% redefines the caption setup of the figures environment in the beamer class.
\usepackage{dsfont}
\usepackage[ngerman,english]{babel}   	% Deutsches Wörterbuch usw.
\usepackage{graphicx}	% wird für Figur Umgebung benötigt
\usepackage{color}	% ermöglicht einbinden von Farben
\usepackage{epstopdf}   % Wandelt .eps Dateien automatisch um
\usepackage{url}	% für URL mit \url{.....}
\usepackage{bbm}	% für einheitsmatrix \mathbbm{1}
\usepackage{listings}	% für code (gnuplot, C etc.)
\usepackage[font=small,labelfont=bf]{caption}       % Optionen für Bild- und Codeunterschriften
\usepackage[hidelinks]{hyperref}                    % damit Links in der PDF anklickbar werden
\usepackage{booktabs}	% bessere Tabellen mit Abstand zur hline
\usepackage{here}
\usepackage[T1]{fontenc}
\usepackage[utf8]{luainputenc}
\usepackage{graphicx}
\usepackage{subfig}
\theoremstyle{break}
\newcommand{\norm}[1]{\left\lVert#1\right\rVert}
\newtheorem{theorem}{Satz}
\newtheorem{lemma}{Lemma}
\newtheorem{defi}{Definition}
\newtheorem{prop}{Proposition}
\newtheorem{erin}{Erinnerung}
\newtheorem{nota}{Notation}
\newtheorem{bsp}{Beispiel}
\newtheorem{beob}{Beobachtung}
\newtheorem{cor}{Korollar}
\newtheorem{bem}{Bemerkung}

\title{A Robust Distance Metric for Deep Metric Learning}
\author{\\ \\ \\Authors (Matriculation Numbers): \\
Ziyang Qiu (3565830) and Thomas Rekers (3489800) \\ \\ \\ \\ \\ Report for Advanced Machine Learning by Prof. Ulrich Köthe \\ \\ \\ \\ \\ \\}
\date{20th September 2019 \\ Heidelberg University}

\begin{document}
\maketitle
\vfill
\thispagestyle{empty}
\cleardoublepage

\tableofcontents
\thispagestyle{empty}
\cleardoublepage

\setcounter{page}{1}
\section{Executive Summary}
Deep metric learning deals with the learning of discriminative features to process classification tasks and image clustering. It is also called similarity learning and gained increasing attention in recent years. The basic idea of deep metric learning is that the image input data is mapped by a neural network into a multidimensional feature space, where similar samples are mapped close to each other and dissimilar samples are mapped more distant to each other. This report is based on the paper Signal-to-Noise Ratio: A Robust Distance Metric for Deep Metric Learning. At first we will summarize the paper by explaining the two different approaches in metric learning, the structure learning and the distant learning, and introducing the robust SNR distance metric, which is based on Signal-to-Noise Ratio (SNR) for measuring the similarity of images. We give explanations regarding the modifications in the loss functions for training models with the SNR metric and discuss the results from the experiments based on Alexnet in the paper. Afterwards we introduce our own experiments based on Alexnet and Resnet and present the differences between these two network models. As explained in the paper the SNR metric can be considered as a relative Euclidian metric under special circumstances. In this case relative Euclidian metric means that the ratio between the Euclidian distance of two samples and the Euclidian distance of one sample to the origin of the feature space is returned. Therefore we decided to design an analogous relative metric not based on the Euclidian metric but on the Mahalanobis metric and call it relative Mahalanobis metric. We summarize the properties of the mahalanobis as described in the paper Distance Metric Learning for Large Margin Nearest Neighbor Classification and we present the resuts of our experiments including the self-designed relative Mahalanobis metric. Finally we discuss our results by comparing the scores of all metrics used in the experiments.

\section{Introduction}
The fundamental idea of metric learning is to map the input data, in most cases a set of images, into a multidimensional feature space. Similar samples are mapped close to each other in the feature space and dissimilar samples are mapped farther apart. This approach can be used both for classification and clustering tasks. While in conventional metric learning the images are mapped linearly into the feature, in deep metric learning we use deep neural network models like Alexnet or Resnet to get a nonlinear embedding of the input data.

\begin{figure}[h]
	\centering
  \includegraphics[width=0.7\textwidth]{figure01.png}
  \caption{\textbf{Figure 1:} The flowchart shows the frame work of DML. A given pair of face images $x_1$ and $x_2$ are mapped into the same feature space as $h_1$ and $h_2$ through deep neural networks, where their similarity is calculated.}
\end{figure}

Because of its great successes in last years deep metric learning is widely applied in face recognition, where the model has to be trained with a very large number of different classes. The contrary approach to deep metric learning is to design a network architecture where as many neurons are in the output layer as different classes are in the dataset. In this case, the model identifies each neuron in the output layer with one class of the dataset and the classification is done by selecting the neuron with the largest output. While this approach becomes problematically in case of a large number of different classes and a small number of samples per class, deep metric learning concepts still show good results. \\
In case of many different classes the number of classes is usually greater than the number of neurons in the output layer in deep metric learning. Therefore we cannot predict the class by identifying each class with one neuron. Instead the classification follows from the distances in the feature space. For example, the k-Nearest-Neighbor algorithm can be chosen to assign a sample to its class. \\
In deep metric learning we can distinguish between two different learning approaches. The first and most common approach is structure learning. Structure learning tries to construct more effective structures for the loss functions. For example, many neural network architectures include sampler functions which deliver a customized batch of samples from the feature space to the loss function. Varying the batch size or increasing the number of negative labeled samples in a batch can be strategies in structure learning. The loss function is designed according to the shape of the batch for the backward training. We will introduce such loss function and the shape of their batches in the preliminaries. It is to notice that all loss functions use metrics to get a semantic distance between an anchor sample and a comparison sample in the feature space. But structure learning approaches just try to optimize the sampling for the loss function or the summation and weights inside of the loss function. The metric is not changed in structure learning but is changed in distance learning, what is the second approach of metric learning. \\
The most common learning processes use the Euclidian metric to calculate the distances of the samples in the feature space. In contrast, distance learning tries to get longer distances between similar and shorter distances between dissimilar sammples in the feature space by choosing a suitable metric inside of the loss function.
\section{Preliminaries}
\subsection{Loss Functions}
As mentioned before optimizing a neural network by tuning the objective function of the training process is part of structure learning. The loss functions introduced in this chapter are based on the Euclidian metric. We will see later that the loss functions for other metrics look similar, but in some cases we have to add a regularization term.
\subsubsection{Contrastive Loss}
The most common structure learning approach is constrastive embedding with its objective function contrastive loss. The idea is to introduce a margin so that dissimilar samples are pushed apart by this given margin. Similar samples are pulled as close as possible to each other. The objective function for contrastive loss is
\begin{align}
L\, = \, \frac{1}{2N}\sum_{(i,j)\in \Gamma}^N\left( yd_{i,j}^2+\left( 1-y\right) \max \left( margin - d_{ij}\right)^2\right)\, ,
\end{align}
where $d_{ij}$ is the Euclidean distance between the sample pairs in the feature space. We have $y \, = \, 1$ if two samples are in the same class and $y \, = \, 0$ otherwise. The $margin$ is the threshold to evaluate whether a pair of samples is in the same class or not. Therefore $d_{ij}$ should be greater than the margin in case of a pair of unsimilar samples and less than the margin in case of two samples of the same class. $\Gamma$ is the set of the samples and $N$ is the size of the set.
\subsubsection{Triplet Loss}
Another common objective function is triplet loss. In triplet loss the loss function operates on a set of triplets, where each triplets consists of an anchor sample, a positive sample and a negative sample. Positive means that the considered sample has the same class as the anchor. Otherwise the sample is negative. The objective function affects that the distance between the anchor and the positive sample is smaller than the distance between the anchor and a negative sample with a margin. The formula for triplet loss is
\begin{align}
L \, = \, \frac{1}{|\Gamma|}\sum_{(i,j,k)\in \Gamma}\max (d_{ij}^2 + m - d_{ik}^2,0)\, ,
\end{align}
where $i$ is the index of the anchor, $j$ the index of the positive sample and $k$ the index of the negative sample. $d$ is the Euclidian metric again and $\Gamma$ the set of the samples.
\subsubsection{N-Pair Loss}
N-pair loss is a generalization of triplet loss by exploiting the possibility to interact with negative samples of all classes. This means that we deal with one positive sample from the same class as the anchor and with $N \, - \, 1$ negative samples, where every negative sample is from a different class of the $N \, - \, 1$ other classes. The formula for N-pair loss is 
\begin{align}
L\, = \, \frac{1}{N}\sum_{i=1}^N\log \left( 1+\sum_{j\neq i}\exp \left( f_i' f_j^+ - f_i' f_i^+\right) \right) \, ,
\end{align}
where $f_i\, = \, f(x_i)$ and $\{(x_i, x_i^+)\}_{i=1}^N$ are $N$ pairs of input samples from $N$ different classes with $x_i$ as anchors and $x_i^+$ as their positive samples. It applies for their labels $y_i$ that $y_i \neq y_j \, \forall \, i\neq j$ and therefore $\{x_j^+, j\neq i\}$ are the negative input samples to $x_i$.
\subsubsection{Lifted Structured Loss}
In lifted structured loss all negative samples are incorporated through training. The positive samples are pulled as close as possible to the anchor and the negative samples are pushed away farther than a given margin. The formula for structured lifted loss is
\begin{align}
L &= \frac{1}{2|P|}\sum_{(i,j)\in P}\max\left(L_{ij}, 0\right)^2 \\
\text{with} \quad L_{ij} \, &= \, \log \left( \sum_{(i,k)\in N} \exp \left( \alpha - d_{ik}\right) + \sum_{(j,l)\in N}\exp \left(\alpha - d_{jl}\right)\right)+d_{ij}\, ,
\end{align}
where $\alpha$ is the margin parameter. $N$ denotes the set of negative pairs and $P$ denotes the set of positive pairs.
\begin{figure}[h]%
  \centering
  \subfloat[][]{\includegraphics[width=0.4\linewidth]{figure02a.png}}%
  \subfloat[][]{\includegraphics[width=0.4\linewidth]{figure02b.png}}%
  \qquad
  \subfloat[][]{\includegraphics[width=0.4\linewidth]{figure02a.png}}%
  \subfloat[][]{\includegraphics[width=0.4\linewidth]{figure02b.png}}%
  \caption{\textbf{Figure 2:} Illustration of (a) contrastive loss, (b) triplet loss, (c) N-pair loss and (d) lifted structured loss. The blue circle is anchor, different shapes indicate different classes.}%
\end{figure}
\subsection{Distance Metrics}
Measuring similarities between pairs of samples have been a research topc for many years. The most well-known distance metric undoubtedly is Euclidian distance, fwhich has been widely used in learning discriminative embeddings. But Euclidean distance metric only measures the distance between paired samples in n-dimensional space, and fail to preserve ghe correlation and improve the robustness of the pairs. Some other distance metrics have been proposed to avoid the weakness of Euclidean distance. One of those is Mahalanobis Distance, which is based on correlations between varaibles by which different patterns can be identified and analyzed. Recently, many new improvements in metric learning have been proposed. To improve the generalization of the learned features, Chen et al. introduced the energy confusion metric to confuse the network metric. Taking the angular relationship between samples into account, Jian Wang et al. introduced an Angular Loss improving the robustness of objective against feature variance. A new framework is also being explored. For example, Xinshao Wang et al. proposed ranked list loss that could better make use of datasets by constructing lists consisting of weighted negative examples and positive examples mined from the training set. Moreover, instead of using distance metrics, Flood Sung et al. managed to teach the Relation Network to larn the metric by itself.
\section{Approach of the Paper}
\subsection{Notations and Technical Terms}
In deep metric learning a neural network can be considered as function $f$ called learning function, that maps input data like image samples into a multidimensional feature space. We write $h\, = \, f(\theta, x)$ where x is an image sample, $h$ is the embedded sample in the feature space and $theta$ represents the parameters and weights of the neural network. Given two features $h_i$ and $h_j$ in the feature space, we consider $h_i$ as anchor feature and $h_j$ as compared feature. Then we call $h_i$ signal and $h_j$ noisy signal and define the noise $n_{ij}$ in $h_i$ and $h_j$ as $n_{ij}\,=\,h_j-h_i$. The signal-to-noise ratio is defined as the ratio of signal vvariance to noise variance. Therefore we can define the SNR between $h_i$ and $h_j$ as
\begin{align}
SNR_{i,j}\,&=\,\frac{\mathrm{Var}(h_i)}{\mathrm{Var}(h_j-h_i)}\,=\,\frac{\mathrm{Var}(h_i)}{\mathrm{Var}(n_{ij})} \\
\text{with}\quad \mathrm{Var}(a)\,&=\,\frac{\sum_{i=1}^n(a_i-\mu)^2}{n} \, ,
\end{align}
where $\mu$ denotes the mean of sample $a$.
\subsection{SNR-Based Metric}
From the perspective of information theory, a greater signal variance represents more useful information. In contrast the variance of noise can be seen as a measure of useless information. As a result, the greater the SNR is, the greater is the ratio of useful information in ratio to useless information. The most important constraint to a distance metric in deep metric learning is that similar samples shall have a short distance, while dissimilar samples shall have a longer distance. Though, the SNR, as basing on the variance on the samples and their differences, assumes an independent Gaussian distribution in every component of the samples, what might be a wrong assumption in some cases. As we found in the SNR considerations, greater SNR indicates similarity and smaller SNR indicates dissimilarity. Therefore we define an SNR based distance measure as the reciprocal of the SNR and call it SNR metric $d_S$:
\begin{align}
d_S(h_i,h_j)\,=\,\frac{1}{SNR_{i,j}}\,=\,\frac{\mathrm{Var}(n_{ij})}{\mathrm{Var}(h_i)}\,.
\end{align}
\begin{figure}[h]
	\centering
  \includegraphics[width=0.7\textwidth]{figure03.png}
  \caption{\textbf{Figure 3:} The curves show the comparison of anchor features and the compared features under different SNR distances.}
\end{figure}
In figure 3 we see a comparison between anchor signals and compared signals, that have different SNR distances to the anchor signals. The 32-dimensional anchor features are in a Gaussian distribution with mean zero and variance one. The compared signals are synthesized by adding Gaussian noises with mean zero and different variances $\sigma^2$ to the anchor features, where $\sigma^2\,=\,\{0.2,0.5,1.0\}$. As we see in the figure, the longer the SNR distances are, the greater the differences between the anchor signals and the compared signals. We notice again that the SNR metric may not be suitable if the anchor samples and compared samples are not in Gaussian distribution or the components of the samples are not independent to each other.
\subsection{Superiority Analysis}
Now we want to compare the SNR distance to the Euclidean distance. Given two samples in an n-dimensional feature space. Then their Euclidean distance is defined as
\begin{align}
d_E(a,b)\, = \, \sqrt{\sum_{i=1}^n(a_i-b_i)^2}\, .
\end{align}
Assuming a zero-mean-distribution in every component of the samples, two samples $h_i$ and $h_j$ in an n-dimensional feature space have the SNR distance
\begin{align}
d_S(h_i, h_j)\,=\,\frac{\mathrm{Var}(n_{ij})}{\mathrm{Var}(h_i)}\,=\,\frac{\sum_{m=1}^n(h_{jm}-h_{im})^2}{\sum_{m=1}^M h_{im}^2}\,=\,\frac{d_E(h_i,h_j)^2}{d_E(h_i)^2}\, ,
\end{align}
where $d_E(h_i)$ denotes the Euclidean distance from $h_i$ to the origin zero of the feature space. We will continue with the assumption of a zero-mean-distribution. This assumption is guaranted by a regularization term that we will introduce later. Figure 4 visualizes the differences between Euclidean and SNR distance. We easily notice that the SNR distance gives us more confidence in the determination whether a pair of samples is similar or not. The Euclidean distance can be seen as an absolute error between the two samples by calculating only the distance from one point to another. On the other hand, the SNR distance can be seen as an relative error between the two samples by taking the distance to the origin of the feature space into account.
\begin{figure}[h]
	\centering
  \includegraphics[width=0.7\textwidth]{figure04.png}
  \caption{\textbf{Figure 4:} Illustration of Euclidean distance in a. Illustration of SNR distance in b.}
\end{figure}
Because the parameters of the neural network are optimized while training the learning function, we observe two forces that influence the mapping of the samples as shown in figure 5. The first force influences the intra-class distances of the samples. As we see in formula ..., by reducing the numerator $d_E(h_i,h_j)$, we also reduce the value of the SNR distance and the return of the objective function. Therefore the distance between samples of the same class decreases while training. The second force influences the inter-class disntaces. Because of the denominator $d_E(h_i)$,  the distance to the mean of all samples increases regardless of the selected loss function. The pulling intra-classes-force and the pushing inter-classes-force effects a distribution in the feature space. Where samples of similar classes form close groups with far distance two the groups of other classes.
\begin{figure}[h]
	\centering
  \includegraphics[width=0.7\textwidth]{figure05.png}
  \caption{\textbf{Figure 5:} Illustration of the constraint force in Euclidean distance and SNR distance.}
\end{figure}
\subsection{SNR Distance and Correlation}
Now we derive the relationship between the SNR distance and the correlation coefficient of two samples in the feature space. Assuming a zero-mean-distribution in every component of the samples and an independence of the noise to the signal feature, we get
\begin{align}
\mathrm{corr}(h_i,h_j)\,&=\,\frac{\mathrm{cov}(h_i,h_j)}{\sqrt{\mathrm{var}(h_i)\mathrm{var}(h_j)}}\,=\,\frac{\mathrm{E}(h_i,h_j)}{\sqrt{\mathrm{var}(h_i)\mathrm{var}(h_j)}} \\
&=\,\frac{\mathrm{E}\left( h_i\left( h_i+n_{ij}\right) \right)}{\sqrt{\mathrm{var}(h_i)\mathrm{var}(h_i+n_{ij})}}\,=\,\frac{\mathrm{E}(h_i)^2}{\sqrt{\mathrm{var}(h_i)\mathrm{var}(h_i+n_{ij})}} \\
&= \,\frac{\mathrm{var}(h_i)}{\sqrt{\mathrm{var}(h_i)^2+\mathrm{var}(h_i)\mathrm{var}(n_{ij})}}\, =\,\frac{1}{\sqrt{1+\frac{\mathrm{var}(n_{ij})}{\mathrm{var}(h_i)}}} \\
&=\,\frac{1}{\sqrt{1+\frac{1}{SNR_{ij}}}}\,=\,\frac{1}{\sqrt{1+d_s(h_i,h_j)}}\, .
\end{align}
As we see in figure 6, the SNR distance is between two samples in the feature space, the greater the correlation coefficient between these two samples.
\begin{figure}[h]
	\centering
  \includegraphics[width=0.7\textwidth]{figure06.png}
  \caption{\textbf{Figure 6:} Illustration of the relationship between SNR distance metric and correlation coefficient of two samples.}
\end{figure}
\subsection{Deep SNR-based Metric Learning}
In this chapter we want to apply the SNR distance metric to various objective functions including contrastive loss, triplet loss, lifted structured loss and N-Pair loss. The learned samples in the feature space are denoted as $h_i\,\in\,(h_1,...,h_n)$. Given an anchor sample $h_i$, its positive sample is denoted as $h_i^+$ and its negative sample is denoted as $h_i^-$. We can formulate the SNR distance of two samples $h_i$ and $h_j$ in the feature space in the following objective functions as
\begin{align}
d_{Sij}\,=\, d_S(h_i,h_j)\,=\,\frac{\mathrm{var}(h_i-h_j)}{\mathrm{var}(h_i)}\, .
\end{align}
To guarantee the zero-mean-distribution of the samples in the feature space, we add a regularization term $\lambda L_r$ to the objective functions. $\lambda$ is a hyperparameter with a small value and $L_r$ is defined as
\begin{align}
L_r\,=\,\frac{1}{N}\sum_{i\in N}|\sum_{m=1}^Mh_{im}|\, ,
\end{align}
where N is the number of considered samples and M the dimension of the feature space.
\subsubsection{DSML(cont)}
Our DSML objective function for SNR-based contrastive embedding is
\begin{align}
L\,=\,\sum_{i=1}^{N_i}d_S\left( h_i,h_i^+\right) +\sum_{j=1}^{N_j}\mathrm{max}\left( \alpha -d_S\left( h_j,h_j^-\right),0\right) + \lambda L_r\, ,
\end{align}
where $N_i$ is the number of positive pairs and $N_j$ is the number of negative pairs. $\alpha$ represents the margin for the negative pairs.
\subsubsection{DSML(tri)}
The objective function for SNR-based triplet embedding can be formulated as
\begin{align}
L\,=\,\sum_{i=1}^N\mathrm{max}\left(d_s\left(h_i,h_i^+\right) - d_s\left( h_i,h_i^-\right)+\alpha,0\right)+\lambda L_r\, .
\end{align}
\subsection{DSML(lifted)}
The SNR-based objective function for lifted loss is
\begin{align}
L\,&=\,\frac{1}{2N}\sum_{(i,j)\in P}\mathrm{max}(J_{ij},0)+\lambda L_r \\
\text{with}\quad J_{ij}&=\mathrm{max}\left(\mathrm{max}_{(i,k)\in N}\left(\alpha - \beta d_{Sik}\right), \mathrm{max}_{(j,l)\in N}\left(\alpha - \beta d_{Sjk}\right)\right)+\beta d_{Sij}\, ,
\end{align}
where $P$ denotes the positive pairs and $N$ denotes the negative pairs. $\alpha$ is the margin and $\beta$ is a hyperparamter that ensures the convergence of the loss.
\subsection{DSML(N-pair)}
For every anchor $x_i$ in N-pair loss, we consider a tuple $\{ x_i, x_1^+, x_2^+, ..., x_N^+\} $ with $x_i^+$ as positive comparison sample and $x_j$ where $j\,\neq\, i$ as negative samples. The idea of N-pair loss is not to calculate distances but to measure similarities which is done by a scalar product. As the SNR distance metric is not even a real metric from Mathematical point of view, we do not have a real scalar product in SNR based calculation. Because of that we have to introduce an SNR-based measure for simularity. Given two samples $h_i$ and $h_j$ in the feature space, we define their similarity $S_ij$ as
\begin{align}
S_{ij}\,=\,\frac{1}{d_{Sij}^2}\, = \, \mathrm{SNR}_{ij}^2 \, =\, \frac{\mathrm{var}(h_i)^2}{\mathrm{var}(h_i-h_j)^2}\, .
\end{align}
Accordingly, our SNR-based objective function for N-pair loss can be defined as
\begin{align}
L\,=\,\frac{1}{N}\sum_{i=1}^N\mathrm{log}\left( 1+\sum_{j\neq i}\mathrm{exp}\left( S_{ij}-S_{ii}\right)\right)+\lambda L_r\, .
\end{align}
\subsection{Deep SNR-based Hashing Learning}
Hashing learning is to learn to encode the image samples to binary code. The binary codes of similar images should have short Hamming distances, and the binary codes of unmatched images have long Hamming distances. \\ \\
To show the generality of SNR metric, authors deployed SNR distance metric to deep hashing learning and proposed Deep SNR-based Hashing methods (DSNRH). The main difference between deep hashing learning and deep metric learning is that the embeddings need to be quantized to binary features in hashing learning. And the similar labels are given as: if two images $i$ and $j$ share at least one label they are similar, otherwise, they are dissimilar.
\section{Experiments of the paper}
\subsection{Experiments on Deep Metric Learning}
\subsubsection{Dataset}
\begin{enumerate}
\item CARS196 dataset: contains 16185 images of 196 car models
\item CUB200-2011 dataset: includes 11788 images of 200 bird species
\item CIFAR10 dataset: contains 60000 32x32 color images of 10 classes. 100 images per class as the testing set, and the rest 59000 images as database set. From the database set, 500 images are randomly selected per class as training set.
\end{enumerate}
\subsubsection{Implementation Details and Evaluation Metrics}
Using Tensorflow, authors built AlexNet for deep metric learning. They replace the fc8 with an embedding layer of M hidden units. And they choose the mini-batch stochastic gradient descent (SGD) with 0.9 momentum as optimizer, and fix the batch size of images as 100, except the N-pair method on CIFAR10 being set to 20. \\ \\
For the clustering tasks, the authors make experiment on CUB200-2011 and CARS196, and use NMI and F1 score to measure the performance of different methods. For image retrieval tasks, they calculate the Recall@K for the experiment results on CUB200-2011 and CARS196, and record the MAP and F1 metric for the experiment results on CIFAR10.
\subsubsection{Results and Analysis}
Table 1 and table 2 show the performance of deep metric learning approaches on CARS196 and CUB200-2011. And we can observe that:
\begin{enumerate}
\item In comparison to deept metric learning combined with Euclidean distance, proposed SNR-based metric improves the performance on all bench-mark datasets.
\item Along the column, the best scores are all achieffed by DSML.
\end{enumerate}
\begin{figure}[h]
	\centering
  \includegraphics[width=0.9\textwidth]{table01.png}
  \caption{\textbf{Table 1:} Results on CARS196 with Alexnet.}
\end{figure}
\begin{figure}[h]
	\centering
  \includegraphics[width=0.9\textwidth]{table02.png}
  \caption{\textbf{Table 2:} Results on CUB200-2011 with Alexnet.}
\end{figure}
\begin{figure}[h]
	\centering
  \includegraphics[width=0.9\textwidth]{table03.png}
  \caption{\textbf{Table 3:} Retrieval results on CIFAR10 with Alexnet.}
\end{figure}
Table 3 shows the results of retrievbal tasks on CIFAR10 with two retrieval strategies: Euclidean ranking and Hamming ranking. Similar to table 1 and table 2, DSML outperforms the related Euclidean-based metric learning methods. \\ \\
Figure 7 shows the t-SNR visualization of the features learned by DSML(cont) and contrastive on CIFAR10. It is obvious that the features learned by DSML(cont) have more clear boundary and discriminative structures, while contrastive loss learned a vague structure.
\begin{figure}[h]
	\centering
  \includegraphics[width=0.9\textwidth]{figure07.png}
  \caption{\textbf{Figure 7:} The t-SNR visualization of features learned by DSML(cont) method and the contrastive method with Euclidean distance on the CIFAR10 dataset.}
\end{figure}
Despite the fact that the assumption on which the SNR distance metric based is not obviously established, the experiment result clearly shows that the SNR distance metric is more powerful than the Euclidean distance metric.
\subsection{Experiment on Hashing Learning}
\subsubsection{Datasets}
\begin{enumerate}
\item CIFAR10: 1000 images per class as the test query set, and the rest images are selected as training set and database set.
\item NUS-WIDE: consisted of 269,648 images associated with 81 tags. 100 images per class as the test query images, and the rest  are used as the training set and database set.
\end{enumerate}
\subsubsection{Implementation Details and  Evaluation Metrics}
In DSNRH, CNN-F network architecture is deployed. Optimizer is mini-batch stochastic  gradient descent with 0.9 momentum. Results are evaluated by MAP@5000.
\subsubsection{Results}
Table 4 shows the performance of proposed DSNRH and other five deep hashing methods, including DPSH, DTSH, DRSCH, DSCH.DDRH. DSNRH has the best performance among other methods. Though DPSH and DTSH are based on the network of the same architecture, their performance is still about 2 $\%$ and about 12 $\%$ less than that of DSNRH on CIFAR10 dataset and NUS-WIDE dataset respectively.
\begin{figure}[h]%
  \centering
  \subfloat[][]{\includegraphics[width=0.45\linewidth]{table04a.png}}%
  \subfloat[][]{\includegraphics[width=0.45\linewidth]{table04b.png}}%
  \caption{\textbf{Table 4:} Deep hashing methods on CIFAR10 and NUS-WIDE.}%
\end{figure}
\subsection{Discussion}
The experiment results on benchmark datasets show that the SNR metric can boost the performance of deep metric learning because it can jointly pull features of the similar samples closer and enlarge the distance of features of inter-class samples. And the results also indicate the generalization of SNR metric in deep hashing learning. In addition, if authors could adopt another state-of-art network like ResNet instead of AlexNet to execute the experiments, the results would be more persuasive. Since there is not only one metric, Euclidean distance, authors should make a comprehensive comparison by taking other distance metrics into comparison. Last but not least, as we have mentioned above, this SNR metric is based on the assumption that all the components of features, no matter from which class, should follow the same distribution. From the results, it seems that SNR metric works pretty well so the assumption may be probably established. Here I will try to infer the distribution that $h_i$ may follow. $h_i$ is the sum of previous layer output, which can be formulated as:
\begin{align}
h_i\, =\, \sum_{k=1}X_k\, ,
\end{align}
where $h_i$ denotes one of the components of features, and $X_k$ is the output value of previous layer. $X_k$ must obey a certain distribution which is most likely the quasi Gaussian distribution because of the Hierarchical structure of neural networks. According to the Liapunov law, we can deduce that every component of $h_i$ should also follow quasi Gaussian distribution. Notice that authors add a regularization to the loss function which can not only constrain the mean of $h_i$ to be very close to zero but also lead to a very small standard deviation of $h_i$. Therefore, the distribution of $h_i$ would be a quasi-Gaussian distribution with a mean very close to zero and small standard deviation. Since the mean and the standard deviation is very small, the difference between mean and standard deviation of different $h_i$ are even smaller so that all $h_i$ approximately follow the same distribution. This could well explain why the SNR metric is so successful.
\subsection{Conclusion}
In this paper, authors proposed a distance metric based on Signal-to-Noise Ratio (SNR), which can better present discriminative features and preserve the correlation coefficient of sample pairs than the Euclidean distance. By simply replacing Euclidean distance with SNR distance, the authors construct deep SNR-based metric learning which shows its superiority to state-of-the-art deep metric learning methods on three benchmarks. As an extension, the authors also deployed SNR metric to hashing learning and the proposed deep SNR-based hashing learning methods achieve an outstanding performance on two benchmarks. Though it is not that persuasive because of the networks used in the experiment and lacking a comprehensive comparison with other more distance metrics, we think at least that the SNR-based distance is a good replacement for the Euclidean distance.
\section{Network Architectures}
While the experiments in the paper are based on AlexNet, our own experiments are based both on AlexNet and on ResNet.
\subsection{AlexNet}
AlexNet is a convolutional neural network architecture which was introduced in the paper \dq ImageNet Classification with Deep Convolutional Neural Networks\dq\, in 2012. AlexNet consists of eight layers. The first five layers are all convolutional layers. In contrast to other convolutional neural network architectures of this time, AlexNet works with overlapping pooling. This means that if a pooling layer consists of a grid of pooling units spaced $s$ pixels apart and the size of the summarized neighborhood centered at the pooling units is of size $z\times z$, then $z$ has to be greater than $s$. In case of AlexNet, that is true by choosing $s\,=\, 2$ and $z\,=\, 3$. The last three layers are fully-connected layers. AlexNet uses rectified linear units as output layers.
\begin{figure}[h]
	\centering
  \includegraphics[width=0.9\textwidth]{figure08.png}
  \caption{\textbf{Figure 8:} All layers of AlexNet.}
\end{figure}
\subsection{ResNet}
ResNet is a neural network architecture which was introduced in the paper \dq Deep Residual Learning for Image Recognition\dq\, in 2015. The basic idea of ResNet is to skip layers in the training process. This is done by so called shortcut connections that deliver an input to a layer which is the activation output of the layer two positions before. For high numbers of layers, ResNet architectures show much better results than AlexNet on benchmarks like CIFAR10.
\section{Mahalanobis-Metric}
\subsection{Definition}
A function $d:\, X\times X\rightarrow \mathbb{R}_0^+$, where $X$ is a vector space, is defined as metric, if $d$ satisfies the following conditions:
\begin{enumerate}
\item triangular inequality:
\begin{align}
d(x_i,x_j)+d(x_j,x_k)\, \geq\, d(x_i,x_k)\quad \forall x_i, x_j, x_k \in X
\end{align}
\item non-negativity:
\begin{align}
d(x_i,x_j)\,\geq\, 0\quad \forall x_i,x_j\in X
\end{align}
\item symmetry:
\begin{align}
d(x_i,x_j)\, =\, d(x_j,x_i)\quad \forall x_i, x_j\in X
\end{align}
\item distinguishability:
\begin{align}
d(x_i,x_j)\, = \, 0\quad \Leftrightarrow\quad x_i\, = \, x_j \quad\forall x_i,x_j\in X
\end{align}
\end{enumerate}
By satisfying all four conditions, the Euclidean distance is a metric. The SNR distance is not a metric, because it is obviously not symmetric for example. If a function satisfies the first three conditions, we call the function pseudometric. \\ \\
Now, we define a function $d_L$ that applies a linear transformation to the input vectors and returns the squared euclidean distance:
\begin{align}
d_L(x_i, x_j)\,=\,\norm{L(x_i-x_j)}_2^2\quad \forall x_i,x_j\in X\, ,
\end{align}
where $L$ is the linear transformation. If $L$ is a matrix of full rank, $d_L$ is a metric. Otherwise, $d_L$ is a pseudometric. \\ \\
As we define $M$ as
\begin{align}
M \, = \, L^T L\, ,
\end{align}
$M$ is a positive semidefinite matrix for every real-valued matrix $L$. We define the Mahalanobis metric $d_M$ as
\begin{align}
d_M\, =\, (x_i - x_j)^T M (x_i-x_j)\quad \forall x_i,x_j\in X\, .
\end{align}
In case of a full-rank-matrix $M$, the Mahalanobis metric is a real metric from Mathematical point of view. In any case, the Mahalanobis metric is a pseudometric.
\subsection{Training the Metric}
The Mahalanobis Metric can be chosen as a distance measure in the loss function and has to be optimized while training the network. The entries of matrix $M$ can be considered as hyperparameters of the network. One common approach for optimizing the metric is to select those entry values for $M$ that increase the samples of the same class in the k-nearest neighborhood. There are also some helpful approaches from Relevant Component Analysis and Convex Optimization to optimize the metric while training.
\subsection{Relative Mahalanobis Distance}
As proven before, the SNR metric can be considered as a relative Euclidean metric, because in case of a zero-mean-distribution, the SNR metric $d_S$ can be written as
\begin{align}
d_S(h_i, h_j)\, =\,\frac{d_E(h_i,h_j)^2}{d_E(h_i, h_i)^2}\, ,
\end{align}
where $d_E$ is the Euclidean distance. The authors of the paper explain the superiority of the SNR distance in their experiments by this fact amongst other things. Therefore, we define our own relative Mahalanobis metric $d_{rM}$ analogous to the relative Euclidean metric as
\begin{align}
d_{rM}\, =\, \frac{d_M(h_i,h_j)}{d_M(h_i,h_i)}\, ,
\end{align}
where $d_M$ is the Mahalanobis metric. If $M$ equals the identity matrix, the Mahalanobis metric already corresponds to the squared Euclidean metric. Thus, we do not use squared metrics in the relative Mahalanobis metric, even though we use squared metrics in the relative Euclidean metric.
\section{Our Experiments}
\subsection{Dataset}
Our dataset is CARS196. It consists of 16,185 images of 196 different car models. The training set contains 8,144 images, the test set contains 8,041 images.
\subsection{Implementation Details}
We used pytorch to implement the neural networks and chose the architecture of Confusezius as basic framework for the training part. The architecture delivers dataloader functions that allow to train the models in customized image batches. Because the loss functions need different batch shapes in the training, we use the preimplemented sampler functions to organize the batches for the loss functions. The loss functions have the metric name as an additional parameter. That allows us to train the same loss function with different metrics. The training is based on $demo.py$ that can be customized via command line parameters. We trained models in AlexNet and ResNet50 architecture. For every architecture we trained with the loss functions triplet loss and lifted loss and trained for all five metrics Euclidean distance $d_E$, SNR distance $d_S$, relative Euclidean distance $d_{rE}$, Mahalanobis distance $d_M$ and relative Mahalanobis distance $d_{rM}$. \\ \\
The evaluation is based on $evaluate.py$ that can also be customized via command line parameters. We implemented Recall@K, MAP@K and F1@K but only calculated Recall@1 and Recall@2 for the comparison to the results from the paper. Usually the recall is determined by calculating the ratio of true positives to true postives and false negatives. Since we do not use our network for direct classifiation, but just project our input data into a multidimensional feature space, we choose a k-nearest-neighbors approach from the paper to calculate the recall. In this approach, Recall@K is calculated by that an anchor scores one in case of a similar sample in the K-nearest neighborhood. Otherwise, the anchor scores zero. We determine the Recall@K by calculating the arithmetic mean of these scores.



\end{document}